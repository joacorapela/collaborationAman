\documentclass[12pt]{article}

\usepackage{graphicx}
\usepackage{url}
\usepackage[colorlinks=true]{hyperref}
\usepackage{endfloat}

\title{Application of the Extended Kalman Filter for Forecasting Mouse
Kinematics and Head Orientation}

\author{Joaqu\'{i}n Rapela}

\begin{document}

\maketitle

\tableofcontents

\section{Introduction}

We used a nonlinear state-space model and the Extended Kalman Filter
algorithm\footnote{\url{https://github.com/joacorapela/lds/blob/master/docs/inference/ekfInference.pdf}}
to infer kinematics and head orientation of simulated
(Section~\ref{sec:simulatedData}) and real mice (Section~\ref{sec:realData}).

\section{Results}

\subsection{Simulated Data}
\label{sec:simulatedData}

Figure~\ref{fig:sim_true} plots the simulated true mouse positions, and
Figure~\ref{fig:sim_measurements} plots the simulated noisy measurements.

\begin{figure}
    \centering
    \href{https://www.gatsby.ucl.ac.uk/~rapela/aman/reports/ekfForKinematicsAndHeadOrientation/figures/34315983_simulation_state_pos2DandHO.html}{\includegraphics[width=5in]{figures/34315983_simulation_state_pos2DandHO.png}}
    \caption{True simulated mouse positions and head orientations. Click on
    the image to get its interactive version, and zoom in the plot to view the
    head orientations (blue arrow).}
    \label{fig:sim_true}
\end{figure}

\begin{figure}
    \centering
    \href{https://www.gatsby.ucl.ac.uk/~rapela/aman/reports/ekfForKinematicsAndHeadOrientation/figures/34315983_simulation_measurement_pos2DandHO.html}{\includegraphics[width=5in]{figures/34315983_simulation_measurement_pos2DandHO.png}}
    \caption{Simulated noisy mouse positions and head orientations. Click on
    the image to get its interactive version, and zoom in the plot to view the
    head orientations (blue arrow).}
    \label{fig:sim_measurements}
\end{figure}

\subsubsection*{Parameter estimation}

Parameter estimation is performed iteratively by maximising the log-likelihood
of the model parameters. Figure~\ref{fig:sim_logLike} plots the obtained
log-likelihood as a function of estimation time. The blue and red traces plot the
log-likelihood obtained during estimation of the kinematics and
head-orientation parameters, respectively.
%
The estimation of the kinematics parameters is much faster since it uses the
standard Kalman filter, while the estimation of the head-orientation parameters
uses the slower extended Kalman filter. The black horizontal line plots the
log-likelihood of the true model parameters.
%
That the log-likelihood of the true model parameter is substantially larger
that obtained by the estimated parameters indicates a problem in the estimation
method that we will address next (Section~\ref{sec:TODO}).

\begin{figure}
    \centering
    \href{https://www.gatsby.ucl.ac.uk/~rapela/aman/reports/ekfForKinematicsAndHeadOrientation/figures/estimationLogLike_simulation.html}{\includegraphics[width=5in]{figures/estimationLogLike_simulation.png}}
    \caption{Log-likelihood of estimated parameters as a function of estimation
    time. The blue and red traces plot the log-likelihood obtained during the
    estimation of the kinematics and head-orientation parameters, respectively.
    The black horizontal line plots the log-likelihood of the true model
    parameters. Click on the image to get its interactive version.}
    \label{fig:sim_logLike}
\end{figure}

Figure~\ref{fig:model_params} plots the true, initial and estimated model
parameters. Figures~\ref{fig:m0_params}
and~\ref{fig:sqrt_diag_V0_params} plot the initial
state mean and standard deviation parameters, respectively.

Except from \texttt{omega\_Q\_std} and most initial state parameters, estimated
parameters accurately approximate their true values.

\begin{figure}
    \centering
    \href{https://www.gatsby.ucl.ac.uk/~rapela/aman/reports/ekfForKinematicsAndHeadOrientation/figures/true_initial_estimated_params_1_modelParams.png}{\includegraphics[width=5in]{figures/true_initial_estimated_params_1_modelParams.png}}

    \caption{True, initial and estimated model parameters.  True and estimated
    model parameters are not displayed for \texttt{cos\_theta\_Q\_std},
    \texttt{sin\_theta\_Q\_std}, \texttt{cos\_theta\_R\_std} and
    \texttt{sin\_theta\_R\_std} are not displayed because they all have small
    absolute value.}

    \label{fig:model_params}
\end{figure}

\begin{figure}
    \centering
    \href{https://www.gatsby.ucl.ac.uk/~rapela/aman/reports/ekfForKinematicsAndHeadOrientation/figures/true_initial_estimated_params_1_m0.png}{\includegraphics[width=5in]{figures/true_initial_estimated_params_1_m0.png}}

    \caption{True, initial and estimated initial state mean parameters.}

    \label{fig:m0_params}
\end{figure}

\begin{figure}
    \centering
    \href{https://www.gatsby.ucl.ac.uk/~rapela/aman/reports/ekfForKinematicsAndHeadOrientation/figures/true_initial_estimated_params_1_sqrt_diag_V0.png}{\includegraphics[width=5in]{figures/true_initial_estimated_params_1_sqrt_diag_V0.png}}

    \caption{True, initial and estimated initial state standard deviation parameters.}

    \label{fig:sqrt_diag_V0_params}
\end{figure}

\subsubsection*{Forecasting}

Figure~\ref{fig:sim_h2_pos} and~\ref{fig:sim_h2_HO} plot the true, measured and forecasted positions
and head orientations of the simulated mouse. Forecasting was performed with an horizon of 2 samples.
%
Figures~\ref{fig:sim_h20_pos}-\ref{fig:sim_h200_HO} show forecastings with horizons of 20 and 200 samples.

Forecastings of positions are good for all horizons used. Forecastings of head
orientation are good for horizons of 2 and 20 samples, but break down for 200
samples.
%
The 95\% confidence intervals appear to be well calibrated, as the true
positions and head orientation angles are covered by the confidence interval
95\% of the time.

\begin{figure}
    \centering
    \href{https://www.gatsby.ucl.ac.uk/~rapela/aman/reports/ekfForKinematicsAndHeadOrientation/figures/33351171_pos_forecasting.html}{\includegraphics[width=5in]{figures/33351171_pos_forecasting.png}}
    \caption{True, measured and forecasted (horizon h=2 samples) horizontal (x) and vertical (y)
    position of the simulated mouse. Click on the image to get its
    interactive version.}
    \label{fig:sim_h2_pos}
\end{figure}

\begin{figure}
    \centering
    \href{https://www.gatsby.ucl.ac.uk/~rapela/aman/reports/ekfForKinematicsAndHeadOrientation/figures/33351171_HO_forecasting.html}{\includegraphics[width=5in]{figures/33351171_HO_forecasting.png}}
    \caption{True, measured and forecasted (horizon h=2 samples) sine and
    cosine of the head-orientation angle
    of the simulated mouse. Click on the image to get its
    interactive version.}
    \label{fig:sim_h2_HO}
\end{figure}

\begin{figure}
    \centering
    \href{https://www.gatsby.ucl.ac.uk/~rapela/aman/reports/ekfForKinematicsAndHeadOrientation/figures/35401086_pos_forecasting.html}{\includegraphics[width=5in]{figures/35401086_pos_forecasting.png}}
    \caption{True, measured and forecasted (horizon h=20 samples) horizontal (x) and vertical (y)
    position of the simulated mouse. Click on the image to get its
    interactive version.}
    \label{fig:sim_h20_pos}
\end{figure}

\begin{figure}
    \centering
    \href{https://www.gatsby.ucl.ac.uk/~rapela/aman/reports/ekfForKinematicsAndHeadOrientation/figures/35401086_HO_forecasting.html}{\includegraphics[width=5in]{figures/35401086_HO_forecasting.png}}
    \caption{True, measured and forecasted (horizon h=20 samples) sine and
    cosine of the head-orientation angle
    of the simulated mouse. Click on the image to get its
    interactive version.}
    \label{fig:sim_h20_HO}
\end{figure}

\begin{figure}
    \centering
    \href{https://www.gatsby.ucl.ac.uk/~rapela/aman/reports/ekfForKinematicsAndHeadOrientation/figures/05075129_pos_forecasting.html}{\includegraphics[width=5in]{figures/05075129_pos_forecasting.png}}
    \caption{True, measured and forecasted (horizon h=200 samples) horizontal (x) and vertical (y)
    position of the simulated mouse. Click on the image to get its
    interactive version.}
    \label{fig:sim_h200_pos}
\end{figure}

\begin{figure}
    \centering
    \href{https://www.gatsby.ucl.ac.uk/~rapela/aman/reports/ekfForKinematicsAndHeadOrientation/figures/05075129_HO_forecasting.html}{\includegraphics[width=5in]{figures/05075129_HO_forecasting.png}}
    \caption{True, measured and forecasted (horizon h=200 samples) sine and
    cosine of the head-orientation angle
    of the simulated mouse. Click on the image to get its
    interactive version.}
    \label{fig:sim_h200_HO}
\end{figure}

\subsection{Real Data}
\label{sec:realData}

Figure~\ref{fig:real_posAndHO} plots the real mouse positions and head
orientations that we forecast below.

\begin{figure}
    \centering
    \href{https://www.gatsby.ucl.ac.uk/~rapela/aman/reports/ekfForKinematicsAndHeadOrientation/figures/M24086_20250203_0_tracking_2025-02-06T10_39_59.html}{\includegraphics[width=5in]{figures/M24086_20250203_0_tracking_2025-02-06T10_39_59.png}}
    \caption{Positions and head orientation of the real mouse used for
    forecasting below.}
    \label{fig:real_posAndHO}
\end{figure}

\subsubsection*{Parameter estimation}

Figure~\ref{fig:realData_logLike} plots the obtained
log-likelihood as a function of estimation time. The blue and red traces plot the
log-likelihood obtained during estimation of the kinematics and
head-orientation parameters, respectively.

\begin{figure}
    \centering
    \href{https://www.gatsby.ucl.ac.uk/~rapela/aman/reports/ekfForKinematicsAndHeadOrientation/figures/estimationLogLike_realData.html}{\includegraphics[width=5in]{figures/estimationLogLike_realData.png}}
    \caption{Log-likelihood of estimated parameters as a function of estimation
    time. The blue and red traces plot the log-likelihood obtained during the
    estimation of the kinematics and head-orientation parameters,
    respectively.}
    \label{fig:realData_logLike}
\end{figure}

Figure~\ref{fig:realData_model_params} plots the initial and estimated model
parameters. Figures~\ref{fig:realData_m0_params}
and~\ref{fig:realData_sqrt_diag_V0_params} plot the initial
state mean and standard deviation parameters, respectively.

Figure~\ref{fig:realData_model_params} shows that the estimated model
parameters are substantially different from the initial parameters, and
Figure~\ref{fig:realData_logLike} indicates that the change in these parameters
increased the log-likelihood of the parameters. Thus, parameter estimation was
beneficial.
%
Note, however, that learning did not change much the initial values of the
mean and standard deviation of the initial state
(Figures~\ref{fig:realData_m0_params}
and~\ref{fig:realData_sqrt_diag_V0_params}).

\begin{figure}
    \centering
    \href{https://www.gatsby.ucl.ac.uk/~rapela/aman/reports/ekfForKinematicsAndHeadOrientation/figures/initial_estimated_params_1_modelParams.html}{\includegraphics[width=5in]{figures/initial_estimated_params_1_modelParams.png}}

    \caption{Initial and estimated model parameters.}

    \label{fig:realData_model_params}
\end{figure}

\begin{figure}
    \centering
    \href{https://www.gatsby.ucl.ac.uk/~rapela/aman/reports/ekfForKinematicsAndHeadOrientation/figures/initial_estimated_params_1_m0.html}{\includegraphics[width=5in]{figures/initial_estimated_params_1_m0.png}}

    \caption{Initial and estimated initial state mean parameters.}

    \label{fig:realData_m0_params}
\end{figure}

\begin{figure}
    \centering
    \href{https://www.gatsby.ucl.ac.uk/~rapela/aman/reports/ekfForKinematicsAndHeadOrientation/figures/initial_estimated_params_1_sqrt_diag_V0.html}{\includegraphics[width=5in]{figures/initial_estimated_params_1_sqrt_diag_V0.png}}

    \caption{Initial and estimated initial state standard deviation parameters.}

    \label{fig:realData_sqrt_diag_V0_params}
\end{figure}

\subsubsection*{Forecasting}

Figure~\ref{fig:real_h2_pos} and~\ref{fig:real_h2_HO} plot the measured and
forecasted positions and head orientations of a real mouse. Forecasting was
performed with an horizon of 2 samples.
%
Figures~\ref{fig:real_h10_pos}-\ref{fig:real_h50_HO} show forecastings with
horizons of 10 and 50 samples.

Forecastings of head orientation are reasonable for all horizons. Forecastings
of position are good for horizons of 2 and 10 samples, but they break down for
the horizon of 50 samples.

\begin{figure}
    \centering
    \href{https://www.gatsby.ucl.ac.uk/~rapela/aman/reports/ekfForKinematicsAndHeadOrientation/figures/32963939_pos_forecasting.html}{\includegraphics[width=5in]{figures/32963939_pos_forecasting.png}}
    \caption{Measured and forecasted (horizon h=2 samples) horizontal (x) and
    vertical (y) position of a real mouse. Click on the image to get its
    interactive version.}
    \label{fig:real_h2_pos}
\end{figure}

\begin{figure}
    \centering
    \href{https://www.gatsby.ucl.ac.uk/~rapela/aman/reports/ekfForKinematicsAndHeadOrientation/figures/32963939_HO_forecasting.html}{\includegraphics[width=5in]{figures/32963939_HO_forecasting.png}}
    \caption{Measured and forecasted (horizon h=2 samples) sine and cosine of
    the head-orientation angle of a real mouse. Click on the image to get
    its interactive version.}
    \label{fig:real_h2_HO}
\end{figure}

\begin{figure}
    \centering
    \href{https://www.gatsby.ucl.ac.uk/~rapela/aman/reports/ekfForKinematicsAndHeadOrientation/figures/43112733_pos_forecasting.html}{\includegraphics[width=5in]{figures/43112733_pos_forecasting.png}}
    \caption{Measured and forecasted (horizon h=10 samples) horizontal (x) and
    vertical (y) position of a real mouse. Click on the image to get its
    interactive version.}
    \label{fig:real_h10_pos}
\end{figure}

\begin{figure}
    \centering
    \href{https://www.gatsby.ucl.ac.uk/~rapela/aman/reports/ekfForKinematicsAndHeadOrientation/figures/43112733_HO_forecasting.html}{\includegraphics[width=5in]{figures/43112733_HO_forecasting.png}}
    \caption{Measured and forecasted (horizon h=10 samples) sine and cosine of
    the head-orientation angle of a real mouse. Click on the image to get
    its interactive version.}
    \label{fig:real_h10_HO}
\end{figure}

\begin{figure}
    \centering
    \href{https://www.gatsby.ucl.ac.uk/~rapela/aman/reports/ekfForKinematicsAndHeadOrientation/figures/85124492_pos_forecasting.html}{\includegraphics[width=5in]{figures/85124492_pos_forecasting.png}}
    \caption{Measured and forecasted (horizon h=50 samples) horizontal (x) and
    vertical (y) position of a real mouse. Click on the image to get its
    interactive version.}
    \label{fig:real_h50_pos}
\end{figure}

\begin{figure}
    \centering
    \href{https://www.gatsby.ucl.ac.uk/~rapela/aman/reports/ekfForKinematicsAndHeadOrientation/figures/85124492_HO_forecasting.html}{\includegraphics[width=5in]{figures/85124492_HO_forecasting.png}}
    \caption{Measured and forecasted (horizon h=50 samples) sine and cosine of
    the head-orientation angle of a real mouse. Click on the image to get
    its interactive version.}
    \label{fig:real_h50_HO}
\end{figure}

\section{TODO list}
\label{sec:TODO}

\begin{enumerate}

    \item Fix the problem in the estimation of parameters of the extended
        Kalman filter model. As shown in Figure~\ref{fig:sim_logLike}, the
        method to estimate parameters of the extended Kalman filter model
        cannot achieve the log likelihood obtained with the true model
        parameters.

    \item Use ONIX recordings to check the accuracy of the kinematics model.

    \item Forecast data from other experimental periods.

    \item Evaluate forecasting performance on less noisy real data.

    \item Compare forecasting performance of the extended Kalman filter model
        with that from other forecasters (e.g., RNN, XFADS and simpler
        forecasters).

    \item Evaluate the forecaster with data from a robot with a camera
        exploring the arena with the 360 degree screen.

    \item Integrate forecaster into Bonsai.ML.

\end{enumerate}

\section{Conclusions}

For simulated data we were able to obtain parameter estimates similar to the
true ones, despite starting the estimation from a distant initial condition and
using substantial noise in the simulation.
%
Yet, we noticed a problem in that the log-likelihood of the estimated model was
substantially below its optimal value. We also observed that the parameters of
the initial values of the state were not changed by the optimisation method.
%
Forecasting with the simulated data was good, with well calibrated confidence
intervals containing the true values with high probability.

For real data the learning method was effective, changing the initial parameter
values to increase the likelihood of the data.
%
Forecasting was reasonable for horizons between 10 samples ($\sim$ 300~msec) and 50
samples ($\sim$ 1.5~sec).

It would be useful to compare the current forecaster with other ones, and to
test it with more realistic data.

\section{Methods}

\subsection{Simulated data}

To perform the simulations we used the script
\href{https://github.com/joacorapela/lds_simulations/blob/master/code/scripts/doSimulateNDSwithGaussianNoise.py}{doSimulateNDSwithGaussianNoise.py}.

Learning was performed by maximising the model parameters likelihood by
gradient ascent using PyTorch. This maximisation was performed in two steps. First we
maximise the likelihood of the kinematic parameters, and then we fixed the
kinematics parameters and maximised the likelihood with respect to the head
orientation parameters. To estimate the kinematic parameters we used the script
\href{https://github.com/joacorapela/lds_simulations/blob/master/code/scripts/doEstimateTorchKinematics.py}{doEstimateTorchKinematics.py},
and to estimate the head orientation parameters we used the script
\href{https://github.com/joacorapela/lds_simulations/blob/master/code/scripts/doEstimateTorchKinematicsHO.py}{doEstimateTorchKinematicsHO.py}.

After learning the model parameters, we performed inference on the model latents
using the script
\href{https://github.com/joacorapela/lds_simulations/blob/master/code/scripts/doEKFilter.py}{doEKFilter.py}

After inferring the model latents, we performed forecasting using the script
\href{https://github.com/joacorapela/lds_simulations/blob/master/code/scripts/doEKForecasting.py}{doEKForecasting.py}.

Figures~\ref{fig:sim_true} and~\ref{fig:sim_measurements} were generated with
the script
\href{https://github.com/joacorapela/lds_simulations/blob/master/code/scripts/doPlotSimulation.py}{doPlotSimulation.py},
Figure~\ref{fig:sim_logLike} was generated with the script
    \href{https://github.com/joacorapela/lds_simulations/blob/master/code/scripts/doPlotEstimationLogLikelihood.py}{doPlotEstimationLogLikelihood.py},
    Figures~\ref{fig:model_params}-\ref{fig:sqrt_diag_V0_params} were generated
    with the script
    \href{https://github.com/joacorapela/lds_simulations/blob/master/code/scripts/doPlotTrueInitialEstimatedParams.py}{doPlotTrueInitialEstimatedParams.py},
and Figures~\ref{fig:sim_h2_pos}-\ref{fig:sim_h200_HO} were generated with the
script
\href{https://github.com/joacorapela/lds_simulations/blob/master/code/scripts/doPlotForecasting.py}{doPlotForecasting.py}.

\subsection{Real data}

We limited the analysis to a section lasting 194~seconds, starting at time
1450~seconds, from file
\texttt{M24086\_20250203\_0\_tracking\_2025-02-06T10\_39\_59.csv}
(Figure~\ref{fig:real_posAndHO}).

As for the simulate data, learning was done by maximising the likelihood of the
model parameters in two steps. First the likelihood of the kinematic parameters
was maximise using the script
\href{https://github.com/joacorapela/lds_tracking_posOriMice/blob/master/code/scripts/doEstimateTorchKinematics.py}{doEstimateTorchKinematics.py},
and then the kinematic parameters were fixed and the head orientation
parameters were optimised using the script
\href{https://github.com/joacorapela/lds_tracking_posOriMice/blob/master/code/scripts/doEstimateTorchKinematicsHO.py}{doEstimateTorchKinematicsHO.py}.

Filtering was done with the script
\href{https://github.com/joacorapela/lds_tracking_posOriMice/blob/master/code/scripts/doEKFkinematicsHO.py}{doEKFkinematicsHO.py}
and forecasting with the script
\href{https://github.com/joacorapela/lds_tracking_posOriMice/blob/master/code/scripts/doEKForecasting.py}{doEKForecasting.py}.

Figure~\ref{fig:real_posAndHO} was generated using the script
\href{https://github.com/joacorapela/lds_tracking_posOriMice/blob/master/code/scripts/doPlotData.py}{doPlotData.py},
Figure~\ref{fig:realData_logLike} was generated with the script
\href{https://github.com/joacorapela/lds_tracking_posOriMice/blob/master/code/scripts/doPlotEstimationLogLikelihood.py}{doPlotEstimationLogLikelihood.py},
Figures~\ref{fig:realData_model_params}-\ref{fig:realData_sqrt_diag_V0_params}
were generated with the script
\href{https://github.com/joacorapela/lds_tracking_posOriMice/blob/master/code/scripts/doPlotInitialEstimatedParams.py}{doPlotInitialEstimatedParams.py},
and Figures~\ref{fig:real_h2_pos}-\ref{fig:real_h50_HO} were generated using the script
\href{https://github.com/joacorapela/lds_tracking_posOriMice/blob/master/code/scripts/doPlotForecasting.py}{doPlotForecasting.py}.

\end{document}
